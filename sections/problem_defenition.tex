% Update the heading to match your theme
\section{Problem Defenition}
\label{problem_defenition}

The steadily developing market of cloud computing services and rapidly expanding use of these
services by the users is both beneficial and detrimental. No doubt they provide an unlimited
access to resources and storage facilities but they make the users prone to various threats from
the insiders and outsiders. The cloud systems by default require the use of Internet in order to
make them work which makes them prone to numerous of threats. The assaults on data from
outside the host can still be controlled with the use of encryption techniques and use of various
softwares to authenticate the users and thus providing them access to the privileged data whereas
assaults by employees are much harder to combat.
A malicious insider is a current or a former employee, contractor or any third party business partner who
\\
\begin{enumerate}

  \item Has or had authorized access to system, data or an organization’s network.

  \item Deliberately misused or exceeded the authorized access in such a manner that affected the confidentiality, integrity or the availability of the information of the organization or several information systems in a negative manner.

\end{enumerate}
\\
A malicious insider is therefore an individual having appropriate access rights to the
information system and thus tends to misuse the privileges he has been granted. Nevertheless
the characterization of the attacker is not straightforward. An example in this regard can be
taken of an employee who has been recently fired or who has been given more workload or
who has been scolded in front of other employees tends to have a vengeance against the
corporation or cloud service he is working or has been working for.
The detection of the malignant insiders is difficult to achieve. A few frameworks have been
contemplated to recognize the insider threats; a few of them utilize the proactive forensics,
honeypots and many different strategies. But the fundamental scheme to avoid these insider
attacks is to have a solid infrastructure.
The most unfavourable imaginable outcome for cloud service suppliers as well as the cloud
users is the fact that a malicious system administrator is working for the cloud service provider.
